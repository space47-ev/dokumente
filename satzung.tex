\documentclass[a4paper]{article}
\usepackage[shortlabels]{enumitem}
\usepackage[ngerman]{babel}
\usepackage{csquotes}
\usepackage{col}
\renewcommand\thesection{§ \arabic{section}}
\setlist[enumerate]{label=(\arabic*)}
\usepackage[utf8]{inputenc}
\usepackage{color}
\usepackage{soul}

\title{Satzung des Space47 e.V.}
\author{Space47}
% \date{\today}
\date{27. Mai 2025}

\begin{document}
\maketitle


\renewcommand\thesection{}
\section{Präambel}

Die Informationsgesellschaft unserer Tage ist ohne Computer nicht mehr denkbar.
Die Einsatzmöglichkeiten der automatisierten Datenverarbeitung und Datenübermittlung bergen Chancen,
aber auch Gefahren für den Einzelnen und für die Gesellschaft.
Informations- und Kommunikationstechnologien verändern das Verhältnis Mensch-Maschine und der Menschen untereinander.
Die Entwicklung zur Informationsgesellschaft erfordert ein neues Menschenrecht auf weltweite, ungehinderte Kommunikation. \\ \\
Der Space47 ist eine Gemeinschaft von Menschen, unabhängig von Alter, Geschlecht, ethnischer Zugehörigkeit und
gesellschaftlicher Stellung, die sich für Informationsfreiheit einsetzt und mit den Auswirkungen von Technologien
auf die Gesellschaft sowie das einzelne Lebewesen beschäftigt und das Wissen um diese Entwicklung fördert.

\renewcommand\thesection{§ \arabic{section}}
\setcounter{section}{0}
\section{Name, Sitz, Geschäftsjahr}
\begin{enumerate}
	\item Der Verein führt den Namen \enquote{Space47}. Der Verein soll in das Vereinsregister eingetragen werden und wird dann um den Zusatz \enquote{e.V.} ergänzt.
	\item Der Verein hat seinen Sitz in Duisburg.
	\item Das Geschäftsjahr ist das Kalenderjahr. Das erste Geschäftsjahr beginnt mit der Eintragung des Vereins in das Vereinsregister und endet am 31.12. des Jahres der Eintragung.
\end{enumerate}

\section{Zweck und Gemeinnützigkeit}
\begin{enumerate}
	\item Der Space47 fördert und unterstützt Vorhaben der Bildung und Volksbildung in Hinsicht neuer technischer Entwicklungen im Sinne der Präambel oder führt diese durch. \\ \\
	Der Vereinszweck soll unter anderem durch folgende Mittel erreicht werden:
	\begin{enumerate}[-]
		\item Regelmäßige öffentliche Treffen und Informationsveranstaltungen.
		\item Veranstaltungen und/oder Förderung von Kongressen, Tagungen und virtueller Zusammenkünfte.
		\item Öffentlichkeitsarbeit und Telepublishing in allen Medien.
		\item Arbeitskreise.
		\item Förderung des schöpferisch–kritischen Umgangs mit Technologie.
	\end{enumerate}

	\item Der Space47 verfolgt ausschließlich und unmittelbar gemeinnützige Zwecke im Sinne des Abschnitts \enquote{steuerbegünstigte Zwecke} der Abgabenordnung 1977 (§ 51 ff AO) in der jeweils gültigen Fassung; er dient ausschließlich und unmittelbar der Volksbildung zum Nutzen der Allgemeinheit. \\ \\
	Er darf keine Gewinne erzielen, er ist selbstlos tätig und verfolgt nicht in erster Linie eigenwirtschaftliche Zwecke. Die Mittel des Space47 werden ausschließlich und unmittelbar zu den satzungsgemäßen Zwecken verwendet. Die Mitglieder erhalten keine Zuwendung aus den Mitteln des Space47. \\ \\
	Die Mitglieder dürfen bei ihrem Ausscheiden oder bei Auflösung des Vereins keine Anteile des Vereinsvermögens erhalten. Niemand darf durch Ausgaben, die dem Zwecke des Space47 fremd sind, oder durch unverhältnismäßig hohe Vergütungen begünstigt werden.
\end{enumerate}

\section{Mitgliedschaft}
\begin{enumerate}
	\item Ordentliche Vereinsmitglieder können natürliche und juristische Personen, Handelsgesellschaften, nicht rechtsfähige Vereine sowie Anstalten und Körperschaften des öffentlichen Rechts werden.
	\item Über die Annahme der Beitrittserklärung entscheidet der Vorstand. Die Mitgliedschaft beginnt mit der Annahme der Beitrittserklärung und der Entrichtung des in der Beitragsordnung festgelegten Beitrages.
	\item Die Ablehnung der Beitrittserklärung durch den Vorstand kann nicht angefochten werden.
	\item Die Mitgliedschaft endet durch Austrittserklärung, durch Tod von natürlichen Personen oder durch Auflösung und Erlöschen von juristischen Personen, Handelsgesellschaften, nicht rechtsfähigen Vereinen sowie Anstalten und Körperschaften des öffentlichen Rechts oder durch Ausschluss; die Beitragspflicht für das laufende Geschäftsjahr bleibt hiervon unberührt.
	\item Der Austritt wird durch eine Willenserklärung in Schrift- oder Textform gegenüber dem Vorstand vollzogen.
	\item Die Mitgliederversammlung kann solche Personen, die sich besondere Verdienste um den Verein oder um die von ihm verfolgten satzungsgemäßen Zwecke erworben haben, zu Ehrenmitgliedern ernennen. Ehrenmitglieder haben alle Rechte eines ordentlichen Mitglieds. Die Beitragsleistungen der Ehrenmitglieder regelt die Beitragsordnung.
	\item Bei berechtigtem Interesse, kann jede natürliche Person, die Mitglied ist, eine Liste aller aktiven Mitglieder beim Vorstand einsehen. Das Nähere regelt eine Vereinsordnung, die von der Mitgliederversammlung beschlossen wird.
\end{enumerate}

\section{Rechte und Pflichten der Mitglieder}
\begin{enumerate}
	\item Die Mitglieder sind berechtigt, die Leistungen des Vereins in Anspruch zu nehmen.
	\item Die Mitglieder sind verpflichtet, die satzungsgemäßen Zwecke des Vereins zu unterstützen und zu fördern. Sie sind verpflichtet, die festgesetzten Beiträge zu zahlen.
\end{enumerate}

\section{Ausschluss eines Mitglieds}
\begin{enumerate}
	\item Ein Mitglied kann durch Beschluss des Vorstandes ausgeschlossen werden, wenn es das Ansehen des Vereins schädigt, seinen Beitragsverpflichtungen nachhaltig nicht nachkommt oder wenn ein sonstiger wichtiger Grund vorliegt. Der Vorstand muss dem auszuschließenden Mitglied den Beschluss in Schrift- oder Textform unter Angabe von Gründen mitteilen und ihm auf Verlangen eine Anhörung gewähren.
	\item Gegen den Beschluss des Vorstandes ist innerhalb einer Frist von zwei Monaten nach Zugang des Ausschließungsbeschlusses die Anrufung der Mitgliederversammlung zulässig. Bis zur Mitgliederversammlung ruht die Mitgliedschaft. Die Mitgliederversammlung entscheidet endgültig über den Ausschluss im ersten Tagesordnungspunkt nach der \enquote{Begrüßung und Eröffnung} sowie dem \enquote{Feststellen der Beschlussfähigkeit}.
\end{enumerate}

\section{Beitrag}
\begin{enumerate}
	\item Der Verein erhebt einen Beitrag (in Euro), der durch jedes Mitglied zu bezahlen ist. Das Nähere regelt eine Beitragsordnung, die von der Mitgliederversammlung beschlossen wird.
	\item Im begründeten Einzelfall kann für ein Mitglied durch Vorstandsbeschluss ein von der Beitragsordnung abweichender Beitrag fest gesetzt werden.
\end{enumerate}

\section{Organe des Vereins}
Die Organe des Vereins sind:
\begin{enumerate}[1.]
	\item die Mitgliederversammlung,
	\item der Vorstand.
\end{enumerate}

\section{Mitgliederversammlung}
\begin{enumerate}
	\item Oberstes Beschlussorgan ist die Mitgliederversammlung. Ihrer Beschlussfassung unterliegen alle in dieser Satzung oder im Gesetz vorgesehenen Gegenständen, insbesondere
	\begin{enumerate}[a)]
		\item die Genehmigung des Finanzberichtes,
		\item die Entlastung des Vorstandes,
		\item die Wahl und die Abberufung der Vorstandsmitglieder,
		\item die Bestellung von Finanzprüfer*innen,
		\item Satzungsänderungen,
		\item die Genehmigung der Beitragsordnung und Vereinsordnung,
		\item die Richtlinie über die Erstattung von Reisekosten und Auslagen,
		\item Beschlüsse über Anträge des Vorstandes und der Mitglieder,
		\item die Ernennung von Ehrenmitgliedern,
		\item die Auflösung des Vereins und die Beschlussfassung über die eventuelle Fortsetzung des aufgelösten Vereins.
	\end{enumerate}
	\item Die ordentliche Mitgliederversammlung findet einmal im Jahr statt. Außerordentliche Mitgliederversammlungen werden auf Beschluss des Vorstandes abgehalten, wenn die Interessen des Vereins dies erfordern, oder wenn mindestens 5 Mitglieder dies unter Angabe des Zwecks und der Gründe in Schrift- oder Textform beantragen. Der Vorstand hat dann innerhalb einer Frist von sechs Wochen die Mitgliederversammlung durchzuführen.

	\item Die Einberufung der Mitgliederversammlung erfolgt in Schrift- oder Textform durch ein Vorstandsmitglied mit einer Frist von mindestens zwei Wochen. Hierbei ist die Tagesordnung bekannt zu geben und es sind alle Informationen zugänglich zu machen. Anträge zur Tagesordnung sind mindestens eine Woche vor der Mitgliederversammlung beim Vorstand einzureichen. Über die Behandlung von Initiativanträgen entscheidet die Mitgliederversammlung.

	\item Beschlüsse über Satzungsänderungen und über die Auflösung des Vereins können nur in einer Mitgliederversammlung gefasst werden, in der diese Tagesordnungspunkte ausdrücklich angekündigt worden sind. Solche Beschlüsse bedürfen zu ihrer Rechtswirksamkeit der Dreiviertelmehrheit der teilnehmenden Mitglieder.

	\item Vorbehaltlich Absatz 4 bedürfen die Beschlüsse einer Mitgliederversammlung der einfachen Mehrheit der Stimmen der teilnehmenden Mitglieder.

	\item Jedes Mitglied hat eine Stimme. Juristische Personen haben einen Stimmberechtigten schriftlich zu bestellen.

	\item Die Mitgliederversammlung wird von dem/der Vorsitzenden, bei seiner Verhinderung von dem/der stellvertretenden Vorsitzenden oder einem anderen vom Vorstand hierzu bestellten Vorstandsmitglied geleitet.

	\item Auf Antrag eines Mitglieds ist geheim abzustimmen. Über die Mitgliederversammlung ist ein Protokoll anzufertigen, das vom Versammlungsleiter und dem Schriftführer zu unterzeichnen ist; das Protokoll ist allen Mitgliedern zugänglich zu machen.

	\item Die Mitgliederversammlung ist beschlussfähig, wenn mindestens ein Viertel der Mitglieder anwesend sind. Die Beschlussunfähigkeit ist festzustellen; bei Beschlussunfähigkeit ist ein zweites Mal einzuladen. Diese Mitgliederversammlung ist dann unabhängig von der Zahl der erschienen Mitglieder beschlussfähig.
\end{enumerate}

\section{Vorstand}
\begin{enumerate}
	\item Der Vorstand besteht aus mindestens vier Mitgliedern, und zwar:
	\begin{enumerate}[a)]
		\item dem/der Vorsitzenden,
		\item einem/einer stellvertretenden Vorsitzenden,
		\item dem/der Kassenwart*in,
		\item dem/der Schriftführer*in.
	\end{enumerate}

	\item Bis zu zwei weitere Vorstandsmitglieder werden bei Bedarf durch die Mitgliederversammlung gewählt.

	\item Vorstand im Sinne des §26 Abs. 2 BGB sind alle Vorstandsmitglieder. Je zwei Vorstandsmitglieder vertreten den Verein gemeinschaftlich. Der Vorstand ist von den Beschränkungen des §181 BGB freigestellt.

	\item Die Amtsdauer der Vorstandsmitglieder beträgt ein Jahr; Wiederwahl ist zulässig. Die gewählten Vorstandsmitglieder bleiben bis zu ihrer Amtsniederlegung oder Neuwahl im Amt.

	\item Besteht der Vorstand aus weniger als vier Mitgliedern, so ist unverzüglich eine außerordentliche Mitgliederversammlung einzuberufen um Nachwahlen durchzuführen.

	\item Beschlüsse des Vorstands werden mit der Mehrheit der Stimmen der an der Beschlussfassung teilnehmenden Vorstandsmitglieder, mindestens zwei, gefasst. Bei Stimmengleichheit gibt die Stimme des/der Vorsitzenden, bei seiner Verhinderung die des/der stellvertretenden Vorsitzenden den Ausschlag.

	\item Ein von der Mitgliederversammlung bestimmtes Vorstandsmitglied überwacht als Kassenwart*in die Haushaltsführung und verwaltet unter Beachtung etwaiger Vorstandsbeschlüsse das Vermögen des Vereins. Es ist auf eine sparsame und wirtschaftliche Haushaltsführung hinzuwirken. Mit Ablauf des Geschäftsjahres stellt er/sie unverzüglich die Abrechnung sowie die fällige Vermögensübersicht und sonstige Unterlagen von wirtschaftlichem Belang den Finanzprüfern des Vereins zur Verfügung.

	\item Die Vorstandsmitglieder sind grundsätzlich ehrenamtlich tätig; sie haben Anspruch auf Erstattung notwendiger Auslagen im Rahmen einer von der Mitgliederversammlung zu beschließenden Richtlinie über die Erstattung von Reisekosten und Auslagen.

	\item Der Vorstand kann einen \enquote{Beirat} einrichten, der für den Verein beratend und unterstützend tätig wird; in den Beirat können auch Nicht-Mitglieder berufen werden.

\end{enumerate}

\section{Finanzprüfer*innen}
\begin{enumerate}
	\item Zur Kontrolle der Haushaltsführung bestellt die Mitgliederversammlung einen oder zwei Finanzprüfer*innen. Nach Durchführung ihrer Prüfung geben sie dem Vorstand Kenntnis von ihrem Prüfungsergebnis und erstatten der Mitgliederversammlung Bericht.
	\item Die Finanzprüfer*innen dürfen dem Vorstand nicht angehören.
	\item Die Finanzprüfer*innen können auch Nicht-Mitglieder sein.
\end{enumerate}

\filbreak
\section{Auflösung des Vereins}
\begin{enumerate}
	\item Bei der Auflösung des Vereins oder bei Wegfall steuerbegünstigter Zwecke fällt das Vereinsvermögen an eine von der Mitgliederversammlung zu bestimmende steuerbegünstigte Körperschaft, die das Vermögen für gemeinnützige Zwecke zu verwenden hat. Sollte die Mitgliederversammlung binnen Jahresfrist nach Auflösung des Vereins oder Wegfall steuerbegünstigter Zwecke keine solche Körperschaft bestimmt haben, fällt das Vereinsvermögen an
	 \\
	 \\
	\enquote{LABOR e.V.}

	[Amtsgericht Bochum VR 3716]
	\\
	\\
	der diese Mittel unmittelbar und ausschließlich für gemeinnützige Zwecke verwenden darf. Sollte dieser Verein bei Auflösung des Vereins nicht oder nicht mehr gemeinnützig sein, fällt das Vereinsvermögen an eine andere von der Mitgliederversammlung zu bestimmende steuerbegünstigte Körperschaft, die das Vermögen für gemeinnützige Zwecke zu verwenden hat.
\end{enumerate}

\renewcommand\thesection{}
\section{}
Satzung vom 27.05.2025 in Kraft getreten durch die Mitgliederversammlung.

\pagebreak
\renewcommand\thesection{}
\section{Mitglieder}
Unterschrieben am 27.05.2025\\
in Reihenfolge der Anwesenheit des Sitzungsprotokolls vom 27.05.2025\\ \\
A. S.\\ \\
S. J.\\ \\
D. S.\\ \\
J. M.\\ \\
C. A.\\ \\
C. B.\\ \\
J. D.\\ \\
F. D.\\ \\
M. H.\\ \\
J. H.\\ \\
L. L.\\ \\
S. K.\\ \\
M. M.\\ \\
R. P.\\ \\
C. S.\\ \\
C. T.\\ \\
M. W.\\ \\

\end{document}
