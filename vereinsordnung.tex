\documentclass[a4paper]{article}
\usepackage[shortlabels]{enumitem}
\usepackage[ngerman]{babel}
\usepackage{csquotes}
\usepackage{color}
\renewcommand\thesection{§ \arabic{section}}
\setlist[enumerate]{label=(\arabic*)}
\usepackage[utf8]{inputenc}

\title{Vereinsordnung des Space47 (Verein in Gründung)}
\author{Space47}
% \date{\today}
\date{18. Juni 2024}

\begin{document}
\maketitle

\section{Selbstverpflichtung}
Jedes Mitglied verpflichtet sich die Interessen des Vereins zu wahren und zu fördern.

\section{Einsicht in die Mitgliederliste}
Bei Anforderung einer Mitgliederliste gemäß Satzung gilt: \\ \\
Kopien, Fotographien oder sonstige Vervielfältigungen, darunter fällt auch die Weitergabe als Datei, sind ausgeschlossen. Ausnahmen kann die Mitgliederversammlung beschließen.

\subsection{Wechsel des Vorstands}
Alle Personen, die das Vorstandsamt niederlegen, oder aus anderen Gründen aus diesem entlassen werden, haben unverzüglich sämtliche Informationen an den neuen Vorstand zu übergeben. Weiter müssen alle bestehenden physischen und digitalen Kopien, wenn diese nicht übergeben wurden, vernichtet werden. hierunter fallen auch eventuell bestehende Zugriffsberechtigungen, digital sowie physisch.

\pagebreak

\section{Ausschlussverfahren}
Kommt es zu einem Ausschlussverfahren so beginnt die Tagesordnung zwingend mit diesen, in der Reihenfolge unveränderlichen Tagesordnungspunkten:
\begin{enumerate}[1.]
    \item Begrüßung und Feststellung der Beschlussfähigkeit
    \item Abstimmung zur Wirksamkeit des Ausschließungsbeschlusses
\end{enumerate}
\begin{verbatim}
ggf. mit Unterpunkten

    2.1 Person A
    2.2 Person B
    2.3 Person C
\end{verbatim}
Im Anschluss können alle weiteren Tagesordnungspunkte und Tagesordnungsergänzungen, wenn nicht anderweitig geregelt, besprochen werden. \\ \\
Bei Bestätigung des Ausschließungsbeschlusses tritt die Wirksamkeit unmittelbar ein. Die ausgeschlossene Person hat sofort die Mitgliederversammlung und auch Räumlichkeiten des Vereins zu verlassen.


\end{document}